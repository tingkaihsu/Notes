\documentclass[12pt]{article}
\usepackage{indentfirst}
\usepackage{braket}
\usepackage{physics}
\usepackage{graphicx}
\usepackage{times}
\usepackage[export]{adjustbox}
\usepackage{listings}
\usepackage{mathcomp}
\usepackage{hyperref}
\usepackage{bm,amsmath}
\usepackage{float}
\usepackage{minted}
\usepackage{bigints}
\title{Title}
\author{Ting-Kai Hsu}
\date{Date}

\begin{document}

\section{Introduction}

First, we shall introduction the two assumptions from Einstein as below:

\begin{itemize}
    \item The laws of physics are preserved for all observers in inertial frame of reference relative to one another. (law of relativity)
    \item The speed of light in vacuum is the same for all observers regardless of their motion.
\end{itemize}

In relativity, we know that there is no more difference between space and time, and we call them altogether as \textit{spacetime}. And we define the spacetime difference between two points is 

\begin{center}
    \[ (\Delta s)^2 = c^2(time\ interval)^2 - (space\ interval)^2\]
    and if we write out small displacement, we'll get
    \[ (ds)^2 = (cdt)^2 - (dx^2 + dy^2 + dz^2) \]
\end{center}

The value of $ds$ will determine the properties of the interval. For instance, when the object is moving in the speed that is slower than that of the light, then $(ds)^2 > 0$ and thus we call the interval is \textit{timelike}; and when $(ds)^2 < 0$ we would call the interval \textit{spacelike}; when $(ds)^2 = 0$, we call it \textit{lightlike}. What is the difference and physical meaning behind this seperation?
\\
\indent
First, if the distance between two events is \textit{timelike}, that is, $(ds)^2 = (cdt)^2 - (dx^2 + dy^2 + dz^2) > 0$, $dt$ cannot be $0$, physically, we say that these two events cannot happen \textbf{at the same time} but can happen \textbf{at the same place}.
\\
\indent
Second, if the distance between two events is \textit{spacelike}, that is, $(ds)^2 = (cdt)^2 - (dx^2 + dy^2 + dz^2) < 0$, $dt$ can be $0$ this time and there must be space interval between these events, in other words, these two events can only happen \textbf{at the same time} but not \textbf{at the same place}.
\\
\indent
Lastly, if the distance betwween two events is \textit{lightlike}, that is, $(ds)^2 = (cdt)^2 - (dx^2 + dy^2 + dz^2) = 0$, we say light can travel between these two events.
\\
\indent
Since the spacetime interval is the geometric properties of events or objects, its value should be preserved in all inertial frame, in other words, $(ds)^2$ is an invariant quantity called \textbf{invariant spacetime interval}. Because in relativity, the concept of time is no longer be seen as an independent varialbe that together with coordinates describe the motion of objects. However, we're eager to find a new concept that makes things easier, so we define a new concept about time, that is, \textbf{proper time}. The meaning of the proper time is that it can be seen as the motion of object has been measureed in \textbf{rest frame}, in other words, we can always choose an inertial frame that the object is at rest in it, and then we call the time in this rest time \textbf{proper time}.
Now assume an object moving with velocity \textbf{v}, and the proper time is denoted as $\tau$
\begin{center}
    \[ (ds)^2 = (ds')^2 \]
    \[ c^2(d\tau)^2 = c^2(dt)^2 - (vdt)^2 \]
\end{center}

And then we'll have 

\begin{center}
    \[d\tau = dt \sqrt{1 - \frac{v^2}{c^2}} = \frac{1}{\gamma} dt \]
\end{center}

\section{Vectors and the Metric Tensor}

In relativity, we use a special notaion to express the operation in this study. For instance, we would view time and position coordinates as same thing, and denote them as $x^{\mu}$, where $x^0 = ct$ is the time coordinate, and $x^1, x^2, x^3$ are the space coordinates. 
And if we consider time together with our position coordinates, then we shall define two different \textbf{vectors} that are useful to us.

\begin{itemize}
    \item covariant vector: $\mathbf{x^{\mu}} = <x^0, x^1, x^2, x^3> = <ct, x, y, z>$
    \item contravariant vector: $\mathbf{x_{\mu}} = <x^0, -x^1, -x^2, -x^3> = <ct, -x, -y, -z>$
\end{itemize}

The transform between them can be achieved by introducing a special tensor called \textbf{Minkowski tensor}, one can adapt +- - - or -+++ convention as one prefer, here we pick the first one. 

\begin{center}
    \[ \eta_{\mu \nu} = 
    \begin{pmatrix}
        1&0&0&0\\
        0&-1&0&0\\
        0&0&-1&0\\
        0&0&0&-1\\
    \end{pmatrix} \]
\end{center}

\begin{center}
    \[ \mathbf{x_{\mu}} = \eta_{\mu \nu} \mathbf{x^{\nu}} \]
\end{center}

Conversely, the inverse relationship is as below:

\begin{center}
    \[ \eta^{\mu \nu} = (\eta_{\mu \nu})^{-1}\]
    \[ \mathbf{x^{\mu}} = \eta^{\mu \nu} \mathbf{x_{\nu}} \]
\end{center}

\section{Action}

First, we consider the motion of \textbf{free particle} so that the potential is not in our discussion. In relativity, time $t$ should be considered in the generalized coordinates, that is, time $t$ become the $n+1$ th coordinate $q_{n+1}$. The action should be scalar (invariant) and the natural choice of the invariant parameter may be the \textbf{proper time of the particle} $\tau$. However, we know that the action should involve generalized coordinate and generalized velocities, and the component of the generalized velocities would obey the following relation,

\begin{center}
    \[ u \cdot u = u^{\nu} u_{\nu} = c^2 \]
\end{center}
 
which is the constraint that makes the generalized velocities dependent.
Therefore, we must choose another invariant parameter. Suppose we pick an arbitrary invariant parameter $\theta$, and the generalized velocities is defined as below

\begin{center}
    \[ x'^{\nu} \equiv \frac{d x^{\nu}}{d \theta}\]
\end{center}

Then Hamilton principle can be written as 

\begin{center}
    \[ \delta S = \int_{\theta_1}^{\theta_2} \Lambda (x^{\nu}, x'^{\nu}) \,d\theta \]
\end{center}

And the Lagrange's equation would become

\begin{center}
    \[ \frac{d}{d \theta} \left( \frac{\partial \Lambda}{\partial x'^{\mu}} \right) - \frac{\partial \Lambda}{\partial x^{\mu}} = 0 \]
\end{center}
Where the Lagrange fuction $\Lambda$ is the world scalar(invariant).
We should try to make connection between the invariant Lagrange function $\Lambda$ and the previous Lagrange fuction $L$. One way is to change the orignal parameter time $t$ to the invariant parameter.

\begin{center}
    \[ \dot{x}^{\mu} = \frac{dx^{\mu}}{dt} = c\frac{dx^{\mu}}{d\theta} \frac{d\theta}{dct} = c\frac{x'^{\mu}}{x'^{0}} \]
    \[ \delta S = \int_{t_1}^{t_2} L (x, \dot{x}, t) \,dt = \int_{\theta_1}^{\theta_2} L( x^{\mu}, c\frac{x'^{\mu}}{x'^{0}} )\ \frac{x'^{0}}{c} \, d\theta \]
\end{center}

Compare with the action we have mentioned previously in this section, we should have

\begin{center}
    \[ \Lambda(x^{\mu}, x'^{\mu}) = \frac{x'^{0}}{c} L(x^{\mu}, c\frac{x'^{\mu}}{x'^{0}}) \]
\end{center}

We could discuss much deeper into this transformation. First, we know that the light speed is not so important and can be set to 1 if we change the units.
$x'^0$ is the derivative of time $t$ with respect to the invariant parameter $\theta$. Since we haven't find out the physical meaning of this invariant parameter, we're not able to discuss the physical meaning of the new Lagrangian here; however, we know that this equation is caused by considering time $t$ as the $n+1$ th of the gerneralized parameter rather than an independent parameter. Yet, there are some mathemetic properties and consequence that we'd be interested in. First, consider scalar transformation, that is, $\Lambda(x^{\mu}, x'^{\mu}) \rightarrow \Lambda(x^{\mu}, \alpha x'^{\mu})$. And then the most beautiful thing happens, by the transformation equation above

\begin{center}
    \[ \Lambda(x^{\mu}, \alpha x'^{\mu}) = \frac{\alpha x'^0}{c} L(x^{\mu}, c\frac{\alpha x'^{\mu}}{\alpha x'^0}) = \alpha \frac{x'^0}{c} L(x^{\mu}, c\frac{x'^{\mu}}{x'^0}) \]
\end{center}

Which is exacly equal to the orignal relativistic Lagrangian multiplied by a scalar, that is, $\alpha \Lambda(x^{\mu}, x'^{\mu})$. From this powerful properties, we can see that the relativistic Lagrangian is a homogeneous function of the generalized velocities in the first degree. It follows from Euler's theorem on homogeneous functions

\begin{center}
    \[ \Lambda = x'^{\mu} \frac{\partial \Lambda}{\partial x'^{\mu}} \]
\end{center}

And because of this properties, we'd have the following equation holds

\begin{center}
    \[ \left[\frac{d}{d \theta} \left( \frac{\partial \Lambda}{\partial x'^{\mu}} \right) - \frac{\partial \Lambda}{\partial x^{\mu}} \right] x'^{\mu} = 0 \]
\end{center}

The proof is as below

\begin{center}
    
\end{center}
\end{document}