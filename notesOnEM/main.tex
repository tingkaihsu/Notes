\documentclass[12pt]{article}
\usepackage{indentfirst}
\usepackage{braket}
\usepackage{physics}
\usepackage{graphicx}
\usepackage{times}
\usepackage[export]{adjustbox}
\usepackage{listings}
\usepackage{mathcomp}
\usepackage{hyperref}
\usepackage{bm,amsmath}
\usepackage{float}
\usepackage{minted}
\title{Title}
\author{Ting-Kai Hsu}
\date{Date}

\begin{document}
\maketitle
\tableofcontents

\section{Electrostatics}


\section{Temp}
When solving the problems finding the potential, one must notice a vital problem that the timing of using formula,

\begin{center}
    \[ V(\mathbf{r}) = \int_{\mathcal{V}} \frac{1}{4 \pi \epsilon_0 }\rho(\mathbf{r}) \, d\tau \]
\end{center}

Notice that in the cases when \textbf{charges tend to infinity}, this formula isn't valid because its derivation assume reference point is at infinity, that is, the assumption of \textbf{zero potential} at infinity, which is not the case.
\end{document}