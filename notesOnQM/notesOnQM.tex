\documentclass[12pt]{article}
\usepackage{graphicx}
\usepackage{times}
\usepackage[export]{adjustbox}
\usepackage{braket}
\usepackage{listings}
\usepackage{mathcomp}
\usepackage{hyperref}
\usepackage{bm,amsmath}
\usepackage{float}
\usepackage{minted}
\usepackage{physics}
\usepackage{amsmath}
\title{Title}
\author{Ting-Kai Hsu}
\date{Date}

\begin{document}

\section{For operator}
The Associative Axiom of outer product.
that is, the outer product acting on a ket can be seen as a constant(complex) multiple on the a ket.
So that the outer product equals to operator.
\\
Something can is left as exercise that is the bra of the operator,
and then we can make use of the outer product.
\\
One should also note that the notion of operator and eigenvalue is important.
\\
And important thing is completeness relation in quantum mechanics.
It must be statisfied so that we are able to describe the physics state or any quantum state as a linear combination of basis vectors.
We can insert the identity operator \textbf{in any place}.
\\
\textbf{Projectoin Operator} is a outer product that can be used on a arbitrary ket.
and the effect that the outer product acting on this ket is that we'll obtain an another ket, as below:
\begin{center}
$\ketbra*{a'}{a'} \cdot \ket*{\alpha} = \ket*{a'} \cdot \braket*{'a'}{\alpha } = \braket*{a'}{\alpha } \ket*{a'}$
\end{center}
And we say $\braket*{a'}{ \alpha}$ is the portion of $\ket*{\alpha}$ parallel to $\ket*{a'}$ 

\section{Matrix Representation}
We must notice \textbf{ket} and \textbf{bra} has the following relation:
\begin{center}
$\braket*{a'}{a''} = \mathit{c}$
\end{center}
where $c$ is a \textbf{complex number}.
And we shall introduce another notation of \textbf{ket} and \textbf{bra}, that is, \textbf{matrix}.
Because the relation between ket and bra, we'll have \textbf{column matrix} to represent \textbf{ket},
and \textbf{row matirx} to represent \textbf{bra}.
The reason that we can represent them by matrix is that the multiplication of \textbf{row matrix} and \textbf{column matrix} is also a complex number.
For instance, \textbf{ket} can be represented by following:
\begin{center}
$\ket*{\alpha } = \begin{pmatrix}
                  \braket*{a^{(1)}}{\alpha }\\
                  \braket*{a^{(2)}}{\alpha }\\
                  \braket*{a^{(3)}}{\alpha }\\
                  \cdot\\
                  \cdot\\
                  \cdot\\
                  \braket*{a^{(n)}}{\alpha }\\
                  \end{pmatrix}$
\end{center}
and we have the representation of \textbf{bra}
\begin{center}
    $\bra*{\beta } = \begin{pmatrix}
                     \braket*{\beta }{a^{(1)}} & \braket*{\beta }{a^{(2)}}& \braket*{\beta }{a^{(3)}} & \cdot\cdot\cdot & \braket*{\beta }{a^{(n)}}
                     \end{pmatrix}$
\end{center}
Also note that we can represent by the \textbf{complex comjuncate} of the element of the matirx above.
\begin{center}
    $\bra*{\beta } = \begin{pmatrix}
                     \braket*{a^{(1)}}{\beta}^{*} & \braket*{a^{(2)}}{\beta}^{*}& \braket*{a^{(3)}}{\beta}^{*}& \cdot\cdot\cdot & \braket*{a^{(3)}}{\beta}^{*}
                     \end{pmatrix}$
\end{center}
\end{document}